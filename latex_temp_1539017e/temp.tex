```latex
\documentclass[11pt,a4paper]{article}

\usepackage{amsmath}
\usepackage{amssymb}
\usepackage{enumitem}
\usepackage{fancyhdr}
\usepackage{lastpage}
\usepackage{geometry}
\usepackage{titling}
\usepackage{tikz}
\usepackage{xcolor}

\geometry{a4paper, margin=2cm}

\pagestyle{fancy}
\fancyhf{}
\renewcommand{\headrulewidth}{0pt}
\fancyfoot[C]{Page \thepage\ of \pageref{LastPage}}

\begin{document}

\begin{center}
\Large\textbf{Oxford Cambridge and RSA}

\vspace{0.5cm}
\includegraphics[width=2cm]{ocr_logo}

\vspace{1cm}
\large\textbf{A LEVEL}\\
\Large\textbf{COMPUTER SCIENCE}

\vspace{0.5cm}
\large\textbf{H446/01 Computer Systems}

\vspace{0.5cm}
\normalsize\textbf{Practice Paper -- Section 1.1: The characteristics of contemporary processors, input, output and storage devices}

\vspace{0.5cm}
\textbf{Time allowed: 45 minutes}
\end{center}

\begin{center}
\fbox{\begin{minipage}{0.9\textwidth}
\textbf{INSTRUCTIONS TO CANDIDATES}
\begin{itemize}
\item Write your name, centre number and candidate number in the boxes above.
\item Use black ink. HB pencil may be used for graphs and diagrams only.
\item Answer all the questions.
\item Write your answer to each question in the space provided.
\item Additional paper may be used if necessary but you must clearly show your candidate number, centre number and question number(s).
\item Do not write in the barcodes.
\end{itemize}

\textbf{INFORMATION FOR CANDIDATES}
\begin{itemize}
\item The number of marks is given in brackets [ ] at the end of each question or part question.
\item The total number of marks for this paper is 40.
\item This document consists of 8 pages.
\end{itemize}
\end{minipage}}
\end{center}

\vspace{1cm}

\begin{tabular}{|p{4cm}|p{6cm}|}
\hline
\textbf{Name} & \\ \hline
\textbf{Centre Number} & \\ \hline
\textbf{Candidate Number} & \\ \hline
\textbf{Date} & \\ \hline
\end{tabular}

\vspace{2cm}

\textbf{EXAMINER'S USE ONLY}

\begin{tabular}{|c|c|c|c|c|c|c|}
\hline
Question & 1 & 2 & 3 & 4 & 5 & Total \\ \hline
Marks & /8 & /10 & /6 & /8 & /8 & /40 \\ \hline
\end{tabular}

\vspace{0.5cm}

\textbf{Grade Boundaries:}
\begin{tabular}{|c|c|c|c|c|c|}
\hline
A* & A & B & C & D & E \\ \hline
36 & 32 & 28 & 24 & 20 & 16 \\ \hline
\end{tabular}

\newpage

\section*{Question 1}
\begin{enumerate}[label=(\alph*)]
    \item Define the term 'clock speed' as used in relation to processors. [2 marks]
    
    \vspace{3cm}
    
    \item A processor has a clock speed of 3.2 GHz. Calculate how many clock cycles can be completed in one second. Show your working. [2 marks]
    
    \vspace{3cm}
    
    \item Explain two factors, other than clock speed, that affect the performance of a processor. [4 marks]
    
    \vspace{6cm}
\end{enumerate}

\newpage

\section*{Question 2}
\begin{enumerate}[label=(\alph*)]
    \item Describe what is meant by a multi-core processor. [2 marks]
    
    \vspace{3cm}
    
    \item Explain how a multi-core processor can improve the performance of a computer system compared to a single-core processor with the same clock speed. [4 marks]
    
    \vspace{5cm}
    
    \item Describe two applications that would particularly benefit from a multi-core processor. Justify your answer. [4 marks]
    
    \vspace{5cm}
\end{enumerate}

\newpage

\section*{Question 3}
\begin{enumerate}[label=(\alph*)]
    \item Describe the purpose of cache memory in a processor. [2 marks]
    
    \vspace{3cm}
    
    \item Explain how the following cache types differ in terms of their purpose and operation:
    \begin{enumerate}[label=(\roman*)]
        \item L1 cache
        \item L2 cache
    \end{enumerate}
    [4 marks]
    
    \vspace{6cm}
\end{enumerate}

\newpage

\section*{Question 4}
\begin{enumerate}[label=(\alph*)]
    \item Describe the purpose of the following components of the Fetch-Decode-Execute cycle:
    \begin{enumerate}[label=(\roman*)]
        \item Program Counter (PC)
        \item Memory Address Register (MAR)
        \item Memory Data Register (MDR)
        \item Accumulator
    \end{enumerate}
    [4 marks]
    
    \vspace{8cm}
    
    \item Explain the stages of the Fetch-Decode-Execute cycle. [4 marks]
    
    \vspace{8cm}
\end{enumerate}

\newpage

\section*{Question 5}
\begin{enumerate}[label=(\alph*)]
    \item Describe the difference between volatile and non-volatile memory. Give one example of each. [4 marks]
    
    \vspace{5cm}
    
    \item A user is deciding between a computer with a traditional Hard Disk Drive (HDD) and one with a Solid State Drive (SSD).
    
    Compare the characteristics of HDDs and SSDs in terms of:
    \begin{enumerate}[label=(\roman*)]
        \item Speed
        \item Durability
    \end{enumerate}
    [4 marks]
    
    \vspace{6cm}
\end{enumerate}

\end{document}
```