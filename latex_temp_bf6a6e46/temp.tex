\documentclass[11pt, a4paper]{article}
\usepackage{amsmath, amssymb, enumitem, array, xcolor, ifthen, fancyhdr}
\usepackage[margin=1in]{geometry}

% Define OCR colors
\definecolor{ocrblue}{RGB}{0, 51, 102}

% Set up fancy headers
\pagestyle{fancy}
\fancyhf{}
\renewcommand{\headrulewidth}{0pt}
\fancyfoot[C]{\thepage}

% Command for creating answer lines
\newcommand{\answerlines}[1]{
  \vspace{0.1cm}
  \foreach \x in {1,...,#1}{
    \noindent\makebox[\linewidth]{\rule{\dimexpr\linewidth-2cm}{0.1pt}}\par\vspace{0.5cm}
  }
}

\begin{document}

\begin{center}
\textcolor{ocrblue}{\large\textbf{Oxford Cambridge and RSA}}

\vspace{1cm}
\textcolor{ocrblue}{\LARGE\textbf{A LEVEL}}

\vspace{0.3cm}
\textcolor{ocrblue}{\LARGE\textbf{Computer Science}}

\vspace{0.3cm}
\textcolor{ocrblue}{\Large\textbf{H446/01 Computer Systems}}

\vspace{0.3cm}
\textcolor{ocrblue}{\large\textbf{Practice Paper - Section 1.1: The characteristics of contemporary processors, input, output and storage devices}}

\vspace{0.5cm}
\textbf{Time allowed: 45 minutes}
\end{center}

\vspace{0.5cm}
\noindent\fbox{
\begin{minipage}{\dimexpr\textwidth-2\fboxsep-2\fboxrule}
\textbf{INSTRUCTIONS TO CANDIDATES}
\begin{itemize}
\item Write your name, centre number and candidate number in the boxes above.
\item Use black ink. HB pencil may be used for graphs and diagrams only.
\item Answer \textbf{all} the questions.
\item Write your answer to each question in the space provided.
\item Additional paper may be used if necessary but you must clearly show your candidate number, centre number and question number(s).
\item Do \textbf{NOT} write in the barcodes.
\end{itemize}

\textbf{INFORMATION FOR CANDIDATES}
\begin{itemize}
\item The number of marks is given in brackets [ ] at the end of each question or part question.
\item The total number of marks for this paper is 40.
\item This document consists of 8 pages.
\item Quality of extended responses will be assessed in questions marked with an asterisk (*).
\end{itemize}
\end{minipage}
}

\vspace{1cm}

\begin{tabular}{|p{4cm}|p{7cm}|}
\hline
\textbf{Centre Number} & \textbf{Candidate Number} \\
\hline
& \\
\hline
\textbf{Candidate Name} & \textbf{Date} \\
\hline
& \\
\hline
\end{tabular}

\vspace{0.5cm}

\begin{center}
\textbf{GRADE BOUNDARIES}

\begin{tabular}{|c|c|c|c|c|c|}
\hline
A* & A & B & C & D & E \\
\hline
36-40 & 32-35 & 28-31 & 24-27 & 20-23 & 16-19 \\
\hline
\end{tabular}
\end{center}

\newpage

\begin{center}
\textbf{SECTION 1.1: THE CHARACTERISTICS OF CONTEMPORARY PROCESSORS, INPUT, OUTPUT AND STORAGE DEVICES}
\end{center}

\begin{enumerate}

\item A computer uses a processor with a Von Neumann architecture.

\begin{enumerate}[label=(\alph*)]
\item Describe two key features of the Von Neumann architecture. [2 marks]
\answerlines{4}

\item Explain what is meant by the fetch-execute cycle. [3 marks]
\answerlines{6}

\item The processor in this computer has an 8-bit data bus, a 12-bit address bus, and a 16-bit register size.
\begin{enumerate}[label=(\roman*)]
\item Calculate the maximum amount of directly addressable memory for this processor. Show your working. [2 marks]
\answerlines{4}

\item Explain why the processor's performance might be limited by having an 8-bit data bus. [2 marks]
\answerlines{4}
\end{enumerate}
\end{enumerate}

\newpage

\item A computer system uses a processor with a clock speed of 3.2 GHz and a 4-stage pipeline.

\begin{enumerate}[label=(\alph*)]
\item Describe what is meant by pipelining in a processor. [2 marks]
\answerlines{4}

\item Calculate the theoretical maximum number of instructions that can be completed per second by this processor. Show your working. [2 marks]
\answerlines{4}

\item Explain one factor that might prevent the processor from achieving this theoretical maximum performance. [2 marks]
\answerlines{4}
\end{enumerate}

\item Modern computers often use multi-core processors.

\begin{enumerate}[label=(\alph*)]
\item Explain what is meant by a multi-core processor. [2 marks]
\answerlines{4}

\item Describe two benefits of using a multi-core processor instead of a single-core processor with the same clock speed. [4 marks]
\answerlines{8}

\item Explain why some applications may not perform better on a multi-core processor. [2 marks]
\answerlines{4}
\end{enumerate}

\newpage

\item A computer system uses CISC and RISC processors for different tasks.

\begin{enumerate}[label=(\alph*)]
\item Compare CISC and RISC processors, describing two differences between them. [4 marks]
\answerlines{8}

\item A mobile device manufacturer is deciding whether to use a CISC or RISC processor in their new smartphone.
\begin{enumerate}[label=(\roman*)]
\item Recommend which type of processor would be most suitable. [1 mark]
\answerlines{2}

\item Justify your recommendation with two reasons. [2 marks]
\answerlines{4}
\end{enumerate}
\end{enumerate}

\item Various storage devices can be used in a computer system.

\begin{enumerate}[label=(\alph*)]
\item Explain the difference between volatile and non-volatile memory, giving one example of each. [4 marks]
\answerlines{8}

\item A user needs to choose between solid-state storage (SSD) and magnetic storage (HDD) for their computer.
\begin{enumerate}[label=(\roman*)]
\item Describe two advantages of solid-state storage compared to magnetic storage. [2 marks]
\answerlines{4}

\item Describe one advantage of magnetic storage compared to solid-state storage. [1 mark]
\answerlines{2}
\end{enumerate}
\end{enumerate}

\newpage

\item[6.] (*) A computer manufacturer is designing a new system for video editing professionals.

Discuss the processor characteristics and storage solutions that would be most suitable for this specialized computer system. In your answer, you should consider:
\begin{itemize}
\item Processor architecture and features
\item Type and configuration of primary storage (RAM)
\item Secondary storage options
\item How these choices would benefit video editing tasks
\end{itemize}

[5 marks]

\answerlines{10}

\end{enumerate}

\end{document}